\documentclass[a4paper,titlepage,12pt]{article}

\usepackage[utf8]{inputenc}
\usepackage[OT2]{fontenc}
\usepackage[serbianc, serbian]{babel}

\title{Implementacija $DP$ procedure za iskaznu logiku}
\author{Irena Vasiljevic1, 1018/2021}
\date{jun 2022.}

\begin{document}

\maketitle

\section{Opis algoritma}

$DP$ procedura je algoritam za ispitivanje zadovoljivosti formule iskazne logike, zasnovan na rezoluciji. Procedura ima svojstvo zaustavljanja, potpunosti i saglasnosti. Sushtinski korak algoritma je eliminacija promenljive, koji je zasnovan na pravilu rezolucije. Pored toga, postupak ukljuchuje i naredne operacije:
\begin{itemize}
	\item propagacija jedinichnih klauza ($unit$ $propagation$);
	\item eliminacija chistih literala ($pure$ $literal$ $elimination$).
\end{itemize}
Takodje, vazhan deo procedure je korak eliminacije tautologichnih klauza, na pochetku procedure i nakon svake iteracije. Ulaz u algoritam je skup klauza, a na izlazu su moguc1a dva scenarija:
\begin{itemize}
	\item dobija se prazan skup klauza (u tom sluchaju algoritam prijavljuje zadovoljivost);
	\item dobija se prazna klauza (u tom sluchaju algoritam prijavljuje nezadovoljivost).
\end{itemize}

\section{Opis implementacije}

Procedura je implementirana u programskom jeziku $C$$+$$+$ i sushtinski prati naredni pseudokod:\\

\newpage

\noindent \textbf{Ulaz}: skup klauza $S$\\
\textbf{Izlaz}: istinitosna vrednost koja odgovara svojstvu zadovoljivosti\\

\noindent $repeat:$\\
\indent $//unit$ $propagation:$\\
\indent $while$ $S$ sadrzhi jedinichnu klauzu $\{l\}$:\\
\indent \indent obrishi $\{l\}$ iz $S$\\
\indent \indent za svaku klauzu $c$ u $S$ koja sadrzhi $-l$:\\
\indent \indent \indent obrishi $-l$ iz $c$\\
\indent $//tautology$ $elimination:$\\
\indent za svaku klauzu $c$ u $S$ koja sadrzhi literal $l$ i njegovu negaciju $-l$:\\
\indent \indent obrishi $c$ iz $S$\\
\indent $//pure$ $literal$ $elimination:$\\
\indent $while$ $S$ sadrzhi chist literal $l$:\\
\indent \indent za svaku klauzu $c$ u $S$ koja sadrzhi $l$:\\
\indent \indent \indent obrishi $c$ iz $S$\\
\indent $//stopping$ $conditions:$\\
\indent $if$ $S$ je prazan:\\
\indent \indent $return$ $true$\\
\indent $if$ $S$ sadrzhi praznu klauzu:\\
\indent \indent $return$ $false$\\
\indent $//variable$ $elimination:$\\
\indent izaberi literal $l$ koji se pojavljuje u oba polariteta u $S$\\
\indent \indent za svaku klauzu $c$ iz $S$ koja sadrzhi $l$ i\\
\indent \indent svaku klauzu $d$ iz $S$ koja sadrzhi $-l$:\\
\indent \indent \indent $//resolve:$\\
\indent \indent \indent dodaj klauzu $(c \backslash \{l\}) \cup (d \backslash \{-l\})$ u $S$\\
\indent \indent za svaku klauzu $c$ iz $S$ koja sadrzhi $l$ ili $-l$:\\
\indent \indent \indent obrishi $c$ iz $S$

\section{Prevodjenje i pokretanje}

Program se prevodi sa: $g$$+$$+$ $-o$ $solve$ $main.cpp$ $solver.cpp$, a pokrec1e se sa: $./solve$. Program na ulazu ochekuje formulu u $DIMACS$ $CNF$ formatu. Na izlazu programa bic1e ispisano $SAT$ ukoliko je formula zadovoljiva, a $UNSAT$ ukoliko nije.

\subsection{$DIMACS$ $CNF$ format}

$DIMACS$ $CNF$ format je tekstualna reprezentacija formule u kon\-juktivnoj normalnoj formi. Linije koje pochinju sa $c$ se smatraju komentarima. Reprezentacija formule pochinje linijom $p$ $cnf$ $broj\_promenljivih$ $broj\_klauza$. Nakon toga, predstavljaju se klauze formule. Svaka klauza se zadaje nizom celih brojeva, koji predstavljaju literale, i zavrshava se sa $0$. Na primer, formula $(x \vee \neg y) \wedge (x \vee y \vee \neg z)$ bi imala narednu reprezentaciju:\\

$p$ $cnf$ $3$ $2$\\
\indent $1$ $-2$ $0$\\
\indent $1$ $2$ $-3$ $0$

\end{document}